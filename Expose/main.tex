\documentclass[a4paper,12pt]{scrartcl}
\author{Anisa Mecheraoui}

\usepackage[ngerman]{babel}
\usepackage[]{csquotes}
\usepackage{titling}
\usepackage{fancyhdr}
\usepackage{graphicx}
\usepackage{float}
\usepackage[margin=2.5cm,includeheadfoot, footskip=80pt]{geometry}
\usepackage[breaklinks]{hyperref}
\usepackage{blindtext}
\usepackage[nospace]{varioref}
\usepackage[backend=biber,style=apa6]{biblatex}
\usepackage{numprint}
\usepackage{pgfplots}
\pgfplotsset{compat=1.16}
\usepackage{booktabs}
\usepackage[justification=centering]{caption}
\usepackage[titles]{tocloft}
\usepackage{tabularx}
\usepackage{setspace}
\setstretch{1.5}
\usepackage{sectsty}
\sectionfont{\fontsize{14}{16.8}\selectfont}
\subsectionfont{\fontsize{12}{14.4}\selectfont}
\usepackage{amsmath}
\usepackage{pgffor}
\usepackage[roman]{parnotes}
\usepackage{xifthen}
\usepackage{units}
\usepackage{xcolor}

%bib referenc
\addbibresource{myBib.bib}
%graphicspath
\graphicspath{{images/}}
\definecolor{ba-blau}{HTML}{093a80}
\setlength{\columnsep}{1cm}
\newlength{\drop}
\newcommand{\myLarge}[1]{{\fontsize{20pt}{24pt}\selectfont\color{ba-blau}#1}}
\newcommand{\pageTitel}[1]{{\fontsize{14pt}{17pt}\selectfont\textbf{\color{ba-blau}#1}}}
\fancypagestyle{style1}{
    \setlength{\headheight}{10pt}
    \pagestyle{fancy}
    \fancyhead{}
    \setlength{\headheight}{2cm}
    \chead{\includegraphics[]{deckbild.jpeg} \\}
    %pagefooter
    \fancyfoot{}
    \fancyfoot[R]{\thepage}
}
\fancypagestyle{Deckblatt}{
    \setlength{\headheight}{10pt}
    \pagestyle{fancy}
    \fancyhead{}
    \setlength{\headheight}{2cm}
    \setlength{\headsep}{80pt}
    \chead{\includegraphics[]{deckbild.jpeg} \\}
    %pagefooter
    \fancyfoot{}
    \fancyfoot[R]{\thepage}
}
\addtokomafont{disposition}{\color{ba-blau}}
\addtokomafont{sectionentry}{\color{ba-blau}}

\begin{document}

\thispagestyle{Deckblatt}
%use different pagestyling
\pagenumbering{gobble}

%% title page
\begin{center}
    \myLarge {Titel der Bachelor Thesis} \\[3ex]
    \includegraphics[width=3cm]{Platzhalter_logo.png} \\[3ex]
    \large{Exposé erstellt im Rahmen des Bachelor Thesis Kolloquiums:
} \\
    \large{DD.MM.YYYY}
\end{center}

\renewcommand{\arraystretch}{2.5}
\begin{table}[h]
    \centering
    \begin{tabularx}{\textwidth}{l c}
        Studiengruppe: &  XX\_XX\_20XX-I \\
        Name Studierender: & Max Mustermann\\
        Anzahl der Wörter: & 0\\
        (inkl. wörtliche Zitate / Fußnoten) & ~ \\
        Anzahl der Wörter: & \\
        (exkl. wörtliche Zitate / Fußnoten) & ~ \\
        Akademischer Gutachter: & Prof. Mustermann\\
        Betrieblicher Gutachter: & Herr Firma\\
        Abgabedatum: & DD.MM.YYYY\\
    \end{tabularx}
\end{table}

\newpage

\pagestyle{style1}
\pageTitel{Gleichbehandlung der Geschlechter}\\
Aus Gründen der besseren Lesbarkeit wird bei Personenbezeichnungen und personenbezogenen Hauptwörtern in diesem Projektbericht die maskuline Form verwendet. Entsprechende Begriffe gelten im Sinne der Gleichbehandlung grundsätzlich für alle Geschlechter. Die verkürzte Sprachform hat nur redaktionelle Gründe und beinhaltet keine Wertung
\newpage
\pagenumbering{Roman}
\setcounter{page}{1}
\tableofcontents
\newpage
\listoffigures
\newpage
\listoftables
\newpage

%stores roman numbercounter
\newcounter{preamblecounterstate}
\setcounter{preamblecounterstate}{\value{page}}

% use arabic numbers for actual content pages
\pagenumbering{arabic}

%includes pages written in the folder section
%\section{Demo-Seite}
Auf dieser Seite befinden sich Umsetzungsbeispiele für häufig benötigte Elemente im Fließtext.\\
%Falls keine Seitenanzahl vorhanden ist  dann:\footcite[Vgl.][]{DemoQuelle}
Demo-Quelle \footcite[Vgl.][\printfield{pages}]{DemoQuelle}.

\begin{center}
    \begin{table}[h]
    \centering
    \begin{tabular}{|c|p{6cm}|}
        \hline
        \textbf{Datum} & \textbf{Aktivitäten} \\
        \hline
        Kebab & 7€ \\
        \hline
        Adana & \begin{itemize}
            \item \textbf{Groß}: 8€
            \item \textbf{Klein}: 6€
        \end{itemize} \\
        \hline
        Köfte & \begin{itemize}
            \item 5 Stück: 8€
            \item 2 Stück: 6€
        \end{itemize}\\
        \hline
        Mercimek Suppe & 3€ \\
        \hline
        Dönerteller & 15€ \\
        \hline
    \end{tabular}
    \captionwithfootnotemark{Beispiel Tabelle.}% Die (verpflichtende) Quellenangabe hierzu wird durch den \footcitetext-Befehl weiter unten gesetzt
    \label{tab:example}
    \end{table}
\end{center}
\footcitetext[Vgl.][\printfield{pages}]{DemoQuelle}

Die Tabelle zeigt den Preis eines Dönertellers, dieser lässt sich wie folgt berechnen:
\begin{equation}
    15 = \sum_{n=1}^{10} \frac{n}{20} + \sum_{k=1}^{5} \frac{2k}{10} - \sum_{i=1}^{3} i
\end{equation}

\newpage
Der folgende Abschnitt könnte hilfreich für eine Ausarbeitung in der Informatik sein.
\begin{figure}[h]
    \begin{lstlisting}
        def hello_world():
            print("Hello, World!")
    \end{lstlisting}
    \captionwithfootnotemark{Ausschnitt aus main.py.}% Die (verpflichtende) Quellenangabe hierzu wird durch den \footnotetext-Befehl weiter unten gesetzt
    \label{fig:meincode}
\end{figure}
\footnotetext{Quelle: Eigene Erstellung}

\subsection{Abkürzungen aus dem Verzeichnis}

Diese \ac{CPU} verarbeitet Daten in einem Takt von mehreren \ac{KHZ}.\\
Das kleine Kind war ganz fasziniert von den \acp{BluM}, und wollte unbedingt eine eigene \ac{BluM} als Spielzeug haben. 
Auf dem Kinderfest waren zahlreiche \acp{BluM} zu sehen, jede bunter und fröhlicher als die andere.\\

Erzwungene Kurzschreibweise: \acs{CPU}\\
Erzwungene Langschreibweise: \acl{CPU}

\newpage
\subsection{Beispiel für jeden Quelltyp}

Buch/Monografie\footcite[Vgl.][\printfield{pages}]{theisen2011}\\
Sammelwerk\footcite[Vgl.][\printfield{pages}]{maier2004}\\
Zeitschriften-/Journalartikel\footcite[Vgl.][\printfield{pages}]{chodorowreich2022loan}\\
Zeitungsartikel\footcite[Vgl.][\printfield{pages}]{dick2012neugierige}\\
Internet\footcite[Vgl.][]{capital2014}\\
Gesetztestext\footnote{Vgl. §433 Abs. 1 Satz 1 BGB}\\
Gerichtsurteil\footcite[Vgl.][\printfield{pages}]{bverfgh1968}\\
öffentliches Dokument\footcite[Vgl.][\printfield{pages}]{eu2022access}\\
internes Dokument\footcite[Vgl.][\printfield{pages}]{abcorganigramm}\\% muss auf separatem Datenträger beigefügt werden
(unvollständige Quellenangaben)\footcite[Vgl.][]{blankmaier}\\
\section{Kontext der Ausarbeitung}
\subsection{Stand der Forschung}
\subsection{Relevanz der Untersuchung}
\section{Erstellung der Bachelor-Thesis}


\subsection{Auswahl der genutzten Quellen}


\subsection{Zeitplan}

Im Folgenden wird der Ablauf der Erstellung der Bachelor-Thesis dargestellt.

\begin{center}
    \begin{table}[h]
    \centering
    \begin{tabular}{|c|p{6cm}|}
        \hline
        \textbf{Zeitraum} & \textbf{Aufgaben} \\
        \hline
        01. bis 14. April 2024 & ... \\
        \hline
        15. bis 30. April 2024 & ...\\
        \hline
        01. bis 22. Mai 2024 & ...\\
        \hline
        23. bis 31. Mai 2024 & ... \\
        \hline
        01. bis 14. Juni 2024 & ... \\
        \hline
    \end{tabular}
    \caption{Zeitplan der Erstellung der Bachelor-Thesis}
    \label{tab:timetable}
    \end{table}
\end{center}
\section{Zielsetzung}


\subsection{Grundlagen}


\subsection{Forschungsfrage}



\newpage

% Set page number to Roman for bibliography
\pagenumbering{Roman}

%reads counter stored earlier
\setcounter{page}{\value{preamblecounterstate}}

\printbibliography
\end{document}