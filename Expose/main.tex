% this document uses paper size A4, default fontsize 12 and properties from the KoMa script scartl class
\documentclass[a4paper,12pt]{scrartcl}

% these packages are often needed but not strictly necessary
%\usepackage{numprint}
%\usepackage{pgfplots}
%\pgfplotsset{compat=1.16}
%\usepackage{amsmath}
%\usepackage{pgffor}
%\usepackage[roman]{parnotes}
%\usepackage{xifthen}

% necessary imports
\usepackage[ngerman]{babel}
\usepackage[]{csquotes}
\usepackage{titling}
\usepackage{fancyhdr}
\usepackage{graphicx}
\usepackage{float}
\usepackage[margin=2.5cm,includeheadfoot, footskip=80pt]{geometry}
\usepackage[breaklinks]{hyperref}
\usepackage{blindtext}
\usepackage[nospace]{varioref}
\usepackage[backend=biber, style=apa,sorting=nyt]{biblatex}
\usepackage{booktabs}
\usepackage[justification=centering]{caption}
\usepackage[titles]{tocloft}
\usepackage{tabularx}
\usepackage{setspace}
\usepackage{sectsty}
\usepackage{units}
\usepackage{xcolor}
\usepackage{parskip}
\usepackage{hyperref}

% set metadata for the PDF
\hypersetup{
    pdfauthor={Your Name},      % Add your name here
    pdftitle={The Title},       % Add the document title here
    pdfsubject={The Subject},   % Add the subject of the document here
    pdfkeywords={Some Keywords} % Add relevant keywords here
}

% location of bib and graphics
\addbibresource{myBib.bib}
\graphicspath{{images/}}

% definitions

\definecolor{ba-blau}{HTML}{093a80}
\newcommand{\documentTitle}[1]{{\fontsize{20pt}{24pt}\selectfont\textbf{\color{ba-blau}#1}}}
\newcommand{\pageTitle}[1]{{\fontsize{14pt}{17pt}\selectfont\textbf{\color{ba-blau}#1}}}
% define command for for using full reference in footnote
\newcommand{\fullfootcite}[1]{
  \footnote{\fullcite{#1}}
}

% styling

% set default line spacing
\setstretch{1.5}

% set section headings fontsizes and line spacing
\sectionfont{\fontsize{14}{16.8}\selectfont\fontfamily{ptm}\selectfont}
\subsectionfont{\fontsize{12}{14.4}\selectfont\fontfamily{ptm}\selectfont}
% color section headings
\addtokomafont{disposition}{\color{ba-blau}}
\addtokomafont{sectionentry}{\color{ba-blau}}

% styling of the title page header and footer
\fancypagestyle{Deckblatt}{
    \setlength{\headheight}{10pt}
    \pagestyle{fancy}
    \fancyhead{}
    \setlength{\headheight}{2cm}
    \setlength{\headsep}{80pt}
    \chead{\includegraphics[]{deckbild.jpeg} \\}
    %pagefooter
    \fancyfoot{}
    \fancyfoot[R]{\thepage}
}
% styling of the default page header and footer
\fancypagestyle{defaultPageStyle}{
    \setlength{\headheight}{10pt}
    \pagestyle{fancy}
    \fancyhead{}
    \setlength{\headheight}{2cm}
    \setlength{\headsep}{40pt}
    \chead{\includegraphics[]{deckbild.jpeg} \\}
    %pagefooter
    \fancyfoot{}
    \fancyfoot[R]{\thepage}
}

% start of document

\begin{document}

% use different pagestyling
\thispagestyle{Deckblatt}
% ignore this page when numbering
\pagenumbering{gobble}

% title page
\begin{center}
    \documentTitle {Titel der Bachelor Thesis} \\[3ex]
    \includegraphics[width=3cm]{Platzhalter_logo.png} \\[3ex]
    \large{Exposé erstellt im Rahmen des Bachelor Thesis Kolloquiums:
    } \\
    \large{DD.MM.YYYY}
\end{center}

\renewcommand{\arraystretch}{2.5}
\begin{table}[h]
    \setlength{\tabcolsep}{32pt}
    \begin{tabularx}{\textwidth}{l l}
        Studiengruppe:                      & XX\_XS2X\_XX     \\
        Name Studierender:                  & Max Mustermann   \\
        Anzahl der Wörter:                  & 0                \\ [-15pt]
        (inkl. wörtliche Zitate / Fußnoten) & ~                \\
        Anzahl der Wörter:                  & 0                \\ [-15pt]
        (exkl. wörtliche Zitate / Fußnoten) & ~                \\
        Akademischer Gutachter:             & Prof. Mustermann \\
        Betrieblicher Gutachter:            & Herr Firma       \\
        Abgabedatum:                        & DD.MM.YYYY       \\
    \end{tabularx}
\end{table}

\newpage

% use default page styling from now on
\pagestyle{defaultPageStyle}

\pageTitle{Gleichbehandlung der Geschlechter}\\
Aus Gründen der besseren Lesbarkeit wird bei Personenbezeichnungen und personenbezogenen Hauptwörtern in diesem Projektbericht die maskuline Form verwendet. Entsprechende Begriffe gelten im Sinne der Gleichbehandlung grundsätzlich für alle Geschlechter. Die verkürzte Sprachform hat nur redaktionelle Gründe und beinhaltet keine Wertung
\newpage

% start page numbering in roman numerals
\pagenumbering{Roman}
\setcounter{page}{1}

\tableofcontents
\newpage
\listoffigures
\newpage
\listoftables
\newpage

% store last roman page number
\newcounter{preamblecounterstate}
\setcounter{preamblecounterstate}{\value{page}}

% use arabic numbers for actual content pages
\pagenumbering{arabic}

% include the pages stored in the section folder
\section{Demo-Seite}
Auf dieser Seite befinden sich Umsetzungsbeispiele für häufig benötigte Elemente im Fließtext.\\
%Falls keine Seitenanzahl vorhanden ist  dann:\footcite[Vgl.][]{DemoQuelle}
Demo-Quelle \footcite[Vgl.][\printfield{pages}]{DemoQuelle}.

\begin{center}
    \begin{table}[h]
    \centering
    \begin{tabular}{|c|p{6cm}|}
        \hline
        \textbf{Datum} & \textbf{Aktivitäten} \\
        \hline
        Kebab & 7€ \\
        \hline
        Adana & \begin{itemize}
            \item \textbf{Groß}: 8€
            \item \textbf{Klein}: 6€
        \end{itemize} \\
        \hline
        Köfte & \begin{itemize}
            \item 5 Stück: 8€
            \item 2 Stück: 6€
        \end{itemize}\\
        \hline
        Mercimek Suppe & 3€ \\
        \hline
        Dönerteller & 15€ \\
        \hline
    \end{tabular}
    \captionwithfootnotemark{Beispiel Tabelle.}% Die (verpflichtende) Quellenangabe hierzu wird durch den \footcitetext-Befehl weiter unten gesetzt
    \label{tab:example}
    \end{table}
\end{center}
\footcitetext[Vgl.][\printfield{pages}]{DemoQuelle}

Die Tabelle zeigt den Preis eines Dönertellers, dieser lässt sich wie folgt berechnen:
\begin{equation}
    15 = \sum_{n=1}^{10} \frac{n}{20} + \sum_{k=1}^{5} \frac{2k}{10} - \sum_{i=1}^{3} i
\end{equation}

\newpage
Der folgende Abschnitt könnte hilfreich für eine Ausarbeitung in der Informatik sein.
\begin{figure}[h]
    \begin{lstlisting}
        def hello_world():
            print("Hello, World!")
    \end{lstlisting}
    \captionwithfootnotemark{Ausschnitt aus main.py.}% Die (verpflichtende) Quellenangabe hierzu wird durch den \footnotetext-Befehl weiter unten gesetzt
    \label{fig:meincode}
\end{figure}
\footnotetext{Quelle: Eigene Erstellung}

\subsection{Abkürzungen aus dem Verzeichnis}

Diese \ac{CPU} verarbeitet Daten in einem Takt von mehreren \ac{KHZ}.\\
Das kleine Kind war ganz fasziniert von den \acp{BluM}, und wollte unbedingt eine eigene \ac{BluM} als Spielzeug haben. 
Auf dem Kinderfest waren zahlreiche \acp{BluM} zu sehen, jede bunter und fröhlicher als die andere.\\

Erzwungene Kurzschreibweise: \acs{CPU}\\
Erzwungene Langschreibweise: \acl{CPU}

\newpage
\subsection{Beispiel für jeden Quelltyp}

Buch/Monografie\footcite[Vgl.][\printfield{pages}]{theisen2011}\\
Sammelwerk\footcite[Vgl.][\printfield{pages}]{maier2004}\\
Zeitschriften-/Journalartikel\footcite[Vgl.][\printfield{pages}]{chodorowreich2022loan}\\
Zeitungsartikel\footcite[Vgl.][\printfield{pages}]{dick2012neugierige}\\
Internet\footcite[Vgl.][]{capital2014}\\
Gesetztestext\footnote{Vgl. §433 Abs. 1 Satz 1 BGB}\\
Gerichtsurteil\footcite[Vgl.][\printfield{pages}]{bverfgh1968}\\
öffentliches Dokument\footcite[Vgl.][\printfield{pages}]{eu2022access}\\
internes Dokument\footcite[Vgl.][\printfield{pages}]{abcorganigramm}\\% muss auf separatem Datenträger beigefügt werden
(unvollständige Quellenangaben)\footcite[Vgl.][]{blankmaier}\\
\section{Kontext der Ausarbeitung}
\subsection{Stand der Forschung}
\subsection{Relevanz der Untersuchung}
\section{Erstellung der Bachelor-Thesis}


\subsection{Auswahl der genutzten Quellen}


\subsection{Zeitplan}

Im Folgenden wird der Ablauf der Erstellung der Bachelor-Thesis dargestellt.

\begin{center}
    \begin{table}[h]
    \centering
    \begin{tabular}{|c|p{6cm}|}
        \hline
        \textbf{Zeitraum} & \textbf{Aufgaben} \\
        \hline
        01. bis 14. April 2024 & ... \\
        \hline
        15. bis 30. April 2024 & ...\\
        \hline
        01. bis 22. Mai 2024 & ...\\
        \hline
        23. bis 31. Mai 2024 & ... \\
        \hline
        01. bis 14. Juni 2024 & ... \\
        \hline
    \end{tabular}
    \caption{Zeitplan der Erstellung der Bachelor-Thesis}
    \label{tab:timetable}
    \end{table}
\end{center}
\section{Zielsetzung}


\subsection{Grundlagen}


\subsection{Forschungsfrage}



\newpage

% set page numbering to roman for bibliography
\pagenumbering{Roman}
% read counter stored earlier
\setcounter{page}{\value{preamblecounterstate}}

\printbibliography[title={Literaturverzeichnis}]

\end{document}