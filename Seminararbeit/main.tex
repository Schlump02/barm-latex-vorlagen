% this document uses paper size A4, default fontsize 12 and properties from the KoMa script scartl class
\documentclass[a4paper, 12pt]{scrartcl}

% necessary imports
\usepackage{lmodern}
\usepackage[T1]{fontenc}
\usepackage[utf8]{inputenc}
\usepackage[ngerman]{babel}
\usepackage{titling}
\usepackage{fancyhdr}
\usepackage{graphicx}
\usepackage{float}
\usepackage[top=1.2cm, left=2.5cm, right=2.5cm, bottom=4cm, includeheadfoot, footskip=40pt]{geometry}
\usepackage{blindtext}
\usepackage[nospace]{varioref}
\usepackage[backend=biber, style=apa6, sorting=none]{biblatex}
\usepackage{booktabs}
\usepackage{caption}
\usepackage[titles]{tocloft}
\usepackage{tabularx}
\usepackage{setspace}
\usepackage{sectsty}
\usepackage{units}
\usepackage[table, xcdraw]{xcolor}
\usepackage{parskip}
\usepackage{hyperref}
\usepackage{listings}
\usepackage{color}
\usepackage[printonlyused]{acronym}
\usepackage[titletoc]{appendix}
\usepackage[all]{nowidow}

% display quotation marks ("") as their german counterparts („“)
\usepackage{csquotes}
\MakeOuterQuote{"}


% set metadata for the PDF
\hypersetup{
    pdfauthor={},                 % Add your names here
    pdftitle={Seminararbeit},     % Add the document title here
    pdfsubject={},                % Add the subject of the document here
    pdfkeywords={},               % Add relevant keywords here
    % attribution required by license CC BY 4.0
    pdfcreator={enabled by https://github.com/Schlump02/barm-latex-vorlagen},
    % link colors
    citebordercolor = {0.333 0.725 0.902},% link to bib in footnotes
    linkbordercolor = {0.333 0.725 0.902},% citation marks
    urlbordercolor = {0.333 0.725 0.902},% URLs
    % or disable colored link borders using:
    %hidelinks,
}

% location of bib and graphics
\addbibresource{myBib.bib}
\graphicspath{{images/}}


% definitions

% define command for quickly creating an indirect citation
\newcommand{\indirectcite}[2][]{
  \ifthenelse{\equal{#1}{}}
  {\footcite[Vgl.][\printfield{pages}]{#2}}% (trailing "%" ensure that the line break is not converted to a space)
  {\footcite[Vgl.][#1]{#2}}%
}
% define command for quickly creating a direct quote citation
\newcommand{\directcite}[2][]{
  \ifthenelse{\equal{#1}{}}
  {\footcite[\printfield{pages}]{#2}}%
  {\footcite[#1]{#2}}%
}

% create captions where the text on the page ends with a footnotemark
\newcommand{\captionwithfootnotemark}[1]{\caption[#1]{#1\footnotemark}}

%define commands to quickly reference tables, figures and appendixes
\newcommand{\abb}[1]{\autoref{fig:#1}}
\newcommand{\tab}[1]{\autoref{tab:#1}}
\newcommand{\anh}[1]{\hyperref[sec:A#1]{Anhang A#1}}

% define command for quickly including an image with a bib entry as its source
\newcommand{\image}[6][]{%
  \vspace{1em}%
  \begin{figure}[H]
    \centering
    \ifthenelse{\equal{#1}{}}
      {\includegraphics[width=0.6\textwidth]{#2}}%
      {\includegraphics[width=#1\textwidth]{#2}}%
    \captionwithfootnotemark{#4}
    \label{fig:#3}
  \end{figure}
  \vspace{-1em}%
  \footcitetext[Vgl.][#6]{#5}
}
% define command for quickly including an image with a custom footnote
\newcommand{\customimage}[5][]{%
  \vspace{1em}%
  \begin{figure}[H]
    \centering
    \ifthenelse{\equal{#1}{}}
      {\includegraphics[width=0.6\textwidth]{#2}}%
      {\includegraphics[width=#1\textwidth]{#2}}%
    \captionwithfootnotemark{#4}
    \label{fig:#3}
  \end{figure}
  \vspace{-1em}%
  \footnotetext{#5}
}

% define command for quickly including a table with a custom footnote
\newcommand{\listedtable}[4]{%
  \vspace{1em}%
  \begin{table}[H]
    \centering
    #1
    \captionwithfootnotemark{#3}
    \label{tab:#2}
  \end{table}
  \vspace{-1em}%
  \footnotetext{#4}
}

% Appendix (Anhang)
\newcommand{\listappendixname}{Anhangsverzeichnis}% name (resulting in headline) is optional
\newlistof{appendices}{app}{\listappendixname}
% counter for Anhang-Subsections
\newcounter{anhangsubsec}
\renewcommand{\theanhangsubsec}{\arabic{anhangsubsec}}
% command to create Anhang-Subsections
\newcommand{\anhangsec}[1]{
   \stepcounter{anhangsubsec}
   \subsection*{A\theanhangsubsec\ #1}% asterisk to suppress automatic numbering
   \label{sec:A\theanhangsubsec}
   %\addcontentsline{toc}{subsection}{\protect\numberline{A\theanhangsubsec} #1}% add subsection headline to table of contents
   \addcontentsline{app}{subsection}{\protect\numberline{A\theanhangsubsec} #1}
}
% do not force all footcite texts to end with a dot
\renewcommand{\bibfootnotewrapper}[1]{\bibsentence#1\addspace}
% place a colon after the name of the author or organisation
\renewcommand{\labelnamepunct}{\addcolon\space}
% improve the german bibliography strings for online references
\DefineBibliographyStrings{ngerman}{retrieved={Abgerufen am}, from={von}, nodate={o.J.},}

% define german name for figure labels
\renewcommand{\cftfigpresnum}{Abbildung }
\settowidth{\cftfignumwidth}{Abbildung 10\quad}
% define german name for table labels
\renewcommand{\cfttabpresnum}{Tabelle }
\settowidth{\cfttabnumwidth}{Tabelle 10\quad}

% create a counter for the numbering in Verzeichnis headlines
\newcounter{verzeichnisnummer}
\renewcommand{\theverzeichnisnummer}{\Roman{verzeichnisnummer}\ }% Counter representation
\setcounter{verzeichnisnummer}{1}% start value

% implement counter in the headlines
\addto\captionsngerman{\renewcommand{\listfigurename}{\theverzeichnisnummer Abbildungsverzeichnis}}
\newcommand{\showlistoffigures}{
  \listoffigures
  \addcontentsline{toc}{section}{\theverzeichnisnummer Abbildungsverzeichnis}
  \stepcounter{verzeichnisnummer}
  \newpage
}
\addto\captionsngerman{\renewcommand{\listtablename}{\theverzeichnisnummer Tabellenverzeichnis}}
\newcommand{\showlistoftables}{
  \listoftables
  \addcontentsline{toc}{section}{\theverzeichnisnummer Tabellenverzeichnis}
  \stepcounter{verzeichnisnummer}
  \newpage
}
\newcommand{\bib}{
  \printbibliography[title={\theverzeichnisnummer Literaturverzeichnis}]
  \addcontentsline{toc}{section}{\theverzeichnisnummer Literaturverzeichnis}
  \newpage
}


% styling

\definecolor{ba-blau}{HTML}{093a80}
\definecolor{gray}{HTML}{646973}
\definecolor{darkgreen}{HTML}{408335}
\definecolor{mauve}{HTML}{A9455D}
\definecolor{blue}{HTML}{1455C0}

% styling for code blocks
% please note that PDF should not be used to distribute code meant for anything but reading
\lstset{
  language=python,% this example uses python code styling
  aboveskip=3mm,
  belowskip=3mm,
  showstringspaces=false,
  columns=flexible,
  basicstyle={\small\ttfamily},
  keywordstyle=\color{blue},
  commentstyle=\color{darkgreen},
  stringstyle=\color{mauve},
  numberstyle=\tiny\color{gray},
  numbers=left,
  breaklines=true,
  breakatwhitespace=true,
  tabsize=3,
  lineskip=1pt,
  % no lines above and below the block:
  %frame=none,
  frame=tb,
  inputencoding=utf8,
  extendedchars=true,
  literate={ä}{{\"a}}1 {ö}{{\"o}}1 {ü}{{\"u}}1 {Ä}{{\"A}}1 {Ö}{{\"O}}1 {Ü}{{\"U}}1 {ß}{{\"s}}1,
}

% set default line spacing
\setstretch{1.5}

% optional: set very high tolerance for whitespace between words before enacting automatic hyphenation (Silbentrennung)
% remove these commands for normal automatic hyphenation
\hyphenpenalty=7000
\exhyphenpenalty=7000
\tolerance=10000

% improve labels for unordered lists
\renewcommand{\labelitemii}{$\circ$}
\renewcommand{\labelitemiii}{-}

% Allow up to 4 levels of sectioning
\setcounter{secnumdepth}{4}
\setcounter{tocdepth}{4}
\newcounter{subsubsubsection}[subsubsection]
\renewcommand{\thesubsubsubsection}{\thesubsubsection.\arabic{subsubsubsection}}
% indent subsubsubsections correctly
\newlength{\subsubsubsectionwidth}
\setlength{\subsubsubsectionwidth}{\linewidth}
\addtolength{\subsubsubsectionwidth}{-3.38em}
% create subsubsubsection command
\newcommand{\subsubsubsection}[1]{%
    \refstepcounter{subsubsubsection}%
    \addcontentsline{toc}{paragraph}{\protect\numberline{\thesubsubsubsection}#1}%
    \paragraph*{\thesubsubsubsection\hspace{1em}\parbox[t]{\subsubsubsectionwidth}{#1}\nopagebreak}%
    \leavevmode\\ [1em]
}

% set section headings fontsizes, font family (Times New Roman) and line spacing
\sectionfont{\fontsize{14}{16.8}\selectfont\fontfamily{ptm}\selectfont}
\subsectionfont{\fontsize{12}{14.4}\selectfont\fontfamily{ptm}\selectfont}
\subsubsectionfont{\fontsize{12}{14.4}\selectfont\fontfamily{ptm}\selectfont}
\paragraphfont{\fontsize{12}{14.4}\selectfont\fontfamily{ptm}\selectfont}
% color section headings
\addtokomafont{disposition}{\color{ba-blau}}
\addtokomafont{sectionentry}{\color{ba-blau}}

% styling of the title page header and footer
\fancypagestyle{Deckblatt}{
    \setlength{\headheight}{10pt}
    \pagestyle{fancy}
    \fancyhead{}
    \setlength{\headheight}{2cm}
    \setlength{\headsep}{80pt}
    \chead{\includegraphics[]{deckbild.jpeg} \\}
}
% styling of the default page header and footer
\fancypagestyle{defaultPageStyle}{
    \setlength{\headheight}{10pt}
    \pagestyle{fancy}
    \fancyhead{}
    \setlength{\headheight}{2cm}
    \setlength{\headsep}{40pt}
    \chead{\includegraphics[]{deckbild.jpeg} \\}
    %pagefooter
    \fancyfoot{}
    \fancyfoot[R]{\thepage}% place page numbers in the lower right corner
    %\fancyfoot[LE,RO]{\thepage}% or use this command to alternate the page number position. See README.md for more info first.
}
% color horizontal lines in header and footer
\renewcommand{\headrule}{\color{gray}\hrule width\headwidth height\headrulewidth \vskip-\headrulewidth}
\renewcommand{\footrule}{\color{gray}\hrule width\headwidth height\footrulewidth \vskip-\footrulewidth}
\renewcommand{\headrulewidth}{0.4pt}

% styling of the document title
\newcommand{\documentTitle}[1]{{\fontsize{20pt}{24pt}\selectfont\textbf{\color{ba-blau}#1}}}
% styling of the document subtitle
\newcommand{\documentSubtitle}[1]{{\fontsize{14pt}{17pt}\selectfont\textbf{\color{ba-blau}#1}}}
%styling of non-chapter page titles
\newcommand{\pageTitle}[1]{{\fontsize{14pt}{17pt}\selectfont\fontfamily{ptm}\textbf{\color{ba-blau}#1}}\\ [1em]}

% place the dots that lead to the page numbers in the table of contents
\renewcommand{\cftsecleader}{\cftdotfill{\cftdotsep}} % for sections
\renewcommand{\cftsubsecleader}{\cftdotfill{\cftdotsep}} % for subsections

% remove indentation from entries of the list of figures, tables and appendices
\setlength{\cftfigindent}{0pt}
\setlength{\cfttabindent}{0pt}

% do not justify table cells by default
\renewcommand{\tabularxcolumn}[1]{>{\raggedright\arraybackslash}p{#1}}

% caption styling
\captionsetup{
    font=footnotesize,% use fontsize 10pt
    justification=centering,% center cations horizontally
    width=0.8\linewidth,% only span max. 80% of the width of a text line
    format=plain,% do not hangindent lines from the label
    % print the label in bold:
    %labelfont=bf,
}

% add some spacing after the number in the footnote
\let\oldfootnote\footnote
\renewcommand{\footnote}[1]{\oldfootnote{\hspace{0.2em}#1}}
\let\oldfootnotetext\footnotetext
\renewcommand{\footnotetext}[1]{\oldfootnotetext{\hspace{0.2em}#1}}


% start of document

\begin{document}

% use different pagestyling
\thispagestyle{Deckblatt}
% ignore this page when numbering
\pagenumbering{gobble}

% title page
\begin{center}
    \documentTitle {Titel der Bachelor Thesis} \\[3ex]
    \includegraphics[width=3cm]{Platzhalter_logo.png} \\[3ex]
    \large{Bachelor Thesis\\ [-10pt]
        zur Erlangung des Abschlusses\\ [-10pt]
        Bachelor of [Arts/Science]\\
        Im Studiengang [...]\\ [-10pt]
        an der Berufsakademie Rhein Main
}
\end{center}

\renewcommand{\arraystretch}{2.5}
\begin{table}[h]
    \setlength{\tabcolsep}{32pt}
    \begin{tabularx}{\textwidth}{l l}
        Vorgelegt von:                      & Max Mustermann   \\ [-15pt]
        Studiengruppe:                      & XX\_XS2X\_XX     \\ [-15pt]
        Matrikelummer:                      & 123456           \\ [-15pt]
        Kontaktdaten:                       & [Straße Hausnr.] \\ [-15pt]
        ~                                   & [PLZ Ort]        \\ [-15pt]
        ~                                   & [E-Mail-Adresse] \\
        Anzahl der Wörter:                  & 0                \\ [-18pt]
        (inkl. wörtliche Zitate / Fußnoten) & ~                \\
        Anzahl der Wörter:                  & 0                \\ [-18pt]
        (exkl. wörtliche Zitate / Fußnoten) & ~                \\
        Akademischer Gutachter:             & Prof. Mustermann \\ [-15pt]
        Betrieblicher Gutachter:            & Herr Firma       \\
        Abgabedatum:                        & DD.MM.YYYY       \\
    \end{tabularx}
\end{table}

% use default page styling from now on
\pagestyle{defaultPageStyle}

\pagetitle{Sperrvermerk}
Das vorliegende Exposé zu einer Bachelor Thesis beinhaltet interne vertrauliche Informationen der XYZ GmbH/AG. Die Weitergabe des Inhaltes dieser Arbeit und eventuell beiliegender Abbildungen, Tabellen und Daten im gesamten oder in Teilen ist grundsätzlich untersagt. Es dürfen keinerlei Kopien oder Abschriften, auch nicht in digitaler Form, gefertigt werden. Ausnahmen bedürfen der schriftlichen Genehmigung durch die XYZ GmbH/AG.
\newpage% remove this if not needed
\pageTitle{Gleichbehandlung der Geschlechter}
Aus Gründen der besseren Lesbarkeit wird bei Personenbezeichnungen und personenbezogenen Hauptwörtern in diesem Dokument die maskuline Form verwendet. Entsprechende Begriffe gelten im Sinne der Gleichbehandlung grundsätzlich für alle Geschlechter. Die verkürzte Sprachform hat nur redaktionelle Gründe und beinhaltet keine Wertung.
\newpage
% Wird ein solcher Hinweis nicht vorangestellt, muss laut den Vorgaben geschlechterinklusive Sprache in der Arbeit genutzt werden.

% start page numbering in roman numerals
\pagenumbering{Roman}
\setcounter{page}{1}
\renewcommand{\footrulewidth}{0.4pt}

\tableofcontents
\newpage

\showlistoffigures

\showlistoftables

% store last roman page number
\newcounter{preamblecounterstate}
\setcounter{preamblecounterstate}{\value{page}}

% use arabic numbers for actual content pages
\pagenumbering{arabic}

% include the text section pages located in the given folder
\section{Demo-Seite}
Auf dieser Seite befinden sich Umsetzungsbeispiele für häufig benötigte Elemente im Fließtext.\\
%Falls keine Seitenanzahl vorhanden ist  dann:\footcite[Vgl.][]{DemoQuelle}
Demo-Quelle \footcite[Vgl.][\printfield{pages}]{DemoQuelle}.

\begin{center}
    \begin{table}[h]
    \centering
    \begin{tabular}{|c|p{6cm}|}
        \hline
        \textbf{Datum} & \textbf{Aktivitäten} \\
        \hline
        Kebab & 7€ \\
        \hline
        Adana & \begin{itemize}
            \item \textbf{Groß}: 8€
            \item \textbf{Klein}: 6€
        \end{itemize} \\
        \hline
        Köfte & \begin{itemize}
            \item 5 Stück: 8€
            \item 2 Stück: 6€
        \end{itemize}\\
        \hline
        Mercimek Suppe & 3€ \\
        \hline
        Dönerteller & 15€ \\
        \hline
    \end{tabular}
    \captionwithfootnotemark{Beispiel Tabelle.}% Die (verpflichtende) Quellenangabe hierzu wird durch den \footcitetext-Befehl weiter unten gesetzt
    \label{tab:example}
    \end{table}
\end{center}
\footcitetext[Vgl.][\printfield{pages}]{DemoQuelle}

Die Tabelle zeigt den Preis eines Dönertellers, dieser lässt sich wie folgt berechnen:
\begin{equation}
    15 = \sum_{n=1}^{10} \frac{n}{20} + \sum_{k=1}^{5} \frac{2k}{10} - \sum_{i=1}^{3} i
\end{equation}

\newpage
Der folgende Abschnitt könnte hilfreich für eine Ausarbeitung in der Informatik sein.
\begin{figure}[h]
    \begin{lstlisting}
        def hello_world():
            print("Hello, World!")
    \end{lstlisting}
    \captionwithfootnotemark{Ausschnitt aus main.py.}% Die (verpflichtende) Quellenangabe hierzu wird durch den \footnotetext-Befehl weiter unten gesetzt
    \label{fig:meincode}
\end{figure}
\footnotetext{Quelle: Eigene Erstellung}

\subsection{Abkürzungen aus dem Verzeichnis}

Diese \ac{CPU} verarbeitet Daten in einem Takt von mehreren \ac{KHZ}.\\
Das kleine Kind war ganz fasziniert von den \acp{BluM}, und wollte unbedingt eine eigene \ac{BluM} als Spielzeug haben. 
Auf dem Kinderfest waren zahlreiche \acp{BluM} zu sehen, jede bunter und fröhlicher als die andere.\\

Erzwungene Kurzschreibweise: \acs{CPU}\\
Erzwungene Langschreibweise: \acl{CPU}

\newpage
\subsection{Beispiel für jeden Quelltyp}

Buch/Monografie\footcite[Vgl.][\printfield{pages}]{theisen2011}\\
Sammelwerk\footcite[Vgl.][\printfield{pages}]{maier2004}\\
Zeitschriften-/Journalartikel\footcite[Vgl.][\printfield{pages}]{chodorowreich2022loan}\\
Zeitungsartikel\footcite[Vgl.][\printfield{pages}]{dick2012neugierige}\\
Internet\footcite[Vgl.][]{capital2014}\\
Gesetztestext\footnote{Vgl. §433 Abs. 1 Satz 1 BGB}\\
Gerichtsurteil\footcite[Vgl.][\printfield{pages}]{bverfgh1968}\\
öffentliches Dokument\footcite[Vgl.][\printfield{pages}]{eu2022access}\\
internes Dokument\footcite[Vgl.][\printfield{pages}]{abcorganigramm}\\% muss auf separatem Datenträger beigefügt werden
(unvollständige Quellenangaben)\footcite[Vgl.][]{blankmaier}\\
\section{Leere Kapitelvorlage}



\subsection{Unterüberschrift}



% set page numbering to roman for bibliography
\pagenumbering{Roman}
% read counter stored earlier
\setcounter{page}{\value{preamblecounterstate}}

% Literaturverzeichnis
\bib

\pagenumbering{gobble}% remove page numbering
\renewcommand{\footrulewidth}{0pt}

\pageTitle{Eidesstattliche Erklärung}
Hiermit erklären wir, dass wir die vorliegende Projektarbeit selbständig verfasst und keine anderen als die angegebenen Quellen und Hilfsmittel benutzt und die aus fremden Quellen direkt oder indirekt übernommenen Gedanken als solche kenntlich gemacht haben. Die Arbeit oder Teile hieraus wurde und wird keiner anderen Stelle oder anderen Person im Rahmen einer Prüfung vorgelegt. Wir versichern zudem, dass keine sachliche Übereinstimmung mit einer im Rahmen eines vorangegangenen Studiums angefertigten Seminar-, Haus-, Diplom- oder Abschlussarbeit sowie Bachelor Thesis besteht.
\\ [1.2em]
%Rödermark, den DD.MM.YYYY% Ort und Datum anpassen
Rödermark, die eingangs genannten Studierenden
\\ [1.2em]
%\includegraphics[width=5cm]{Platzhalter_Unterschrift.png}
% Schicker ist es natürlich, ein Bild der eigenen Unterschriften hier einzufügen (genau genommen ist dies auch so vorgegeben)

\end{document}