% Das Abkürzungsverzeichnis benutzt das package acronym
% hier ist die Doku https://www.ctan.org/pkg/acronym
% Nur im Text verwendete Abkürzungen werden angezeigt.
% 2x Kompilieren für Aktualisierung des Abkürzungsverzeichnisses.

% Überschrift
\section*{\theverzeichnisnummer Abkürzungsverzeichnis}
\addcontentsline{toc}{section}{\theverzeichnisnummer Abkürzungsverzeichnis}
\stepcounter{verzeichnisnummer}

% Wichtig: die Abkürzungen müssen in alphabetischer Reihenfolge eingetragen werden

\begin{acronym}[UN] % In den eckigen Klammern das längste Acronym schreiben, dies bestimmt dann das Spacing
  \acro{IT}{Informationstechnologie}
  % Falls eine Abkürzung in der Mehrzahl nicht einfach auf "s" endet, kann sie zusätzlich definiert werden:
  \acroplural{IT}[IT]{Informationstechnologien}
  \acro{UN}{Vereinte Nationen}
\end{acronym}

% Zur Verwendung siehe die demoSection.tex

\newpage