% this document uses paper size A4, default fontsize 12 and creates toc entries for lists
% also uses properties from the KoMa script scrarctl class
\documentclass[a4paper, 12pt, listof=totoc]{scrartcl}

% necessary imports
\usepackage{lmodern}
\usepackage[T1]{fontenc}
\usepackage[utf8]{inputenc}
\usepackage[ngerman]{babel}
\usepackage{titling}
\usepackage{fancyhdr}
\usepackage{graphicx}
\usepackage{float}
\usepackage[top=1.2cm, left=2.5cm, right=2.5cm, bottom=4cm, includeheadfoot, footskip=40pt]{geometry}
\usepackage{blindtext}
\usepackage[nospace]{varioref}
\usepackage[backend=biber, style=apa6, sorting=nyt]{biblatex}
\usepackage{booktabs}
\usepackage{caption}
\usepackage[titles]{tocloft}
\usepackage{tabularx}
\usepackage{setspace}
\usepackage{sectsty}
\usepackage{units}
\usepackage[table, xcdraw]{xcolor}
\usepackage{parskip}
\usepackage{hyperref}
\usepackage{listings}
\usepackage{color}
\usepackage[printonlyused]{acronym}
\usepackage[titletoc]{appendix}
\usepackage[all]{nowidow}

% display quotation marks ("") as their german counterparts („“)
\usepackage{csquotes}
\MakeOuterQuote{"}


% set metadata for the PDF
\hypersetup{
    pdfauthor={},                 % Add your name here
    pdftitle={Bachelor Thesis},   % Add the document title here
    pdfsubject={},                % Add the subject of the document here
    pdfkeywords={},               % Add relevant keywords here
    % attribution required by license CC BY 4.0
    pdfcreator={enabled by https://github.com/Schlump02/barm-latex-vorlagen},
    % link colors
    citebordercolor = {0.333 0.725 0.902},% link to bib in footnotes
    linkbordercolor = {0.333 0.725 0.902},% citation marks
    urlbordercolor = {0.333 0.725 0.902},% URLs
    % or disable colored link borders using:
    %hidelinks,
}

% location of bib and graphics
\addbibresource{Quellen.bib}
\graphicspath{{images/}}


% definitions

% define command for quickly creating an indirect citation
\newcommand{\indirectcite}[2][]{
  \ifthenelse{\equal{#1}{}}
  {\footcite[Vgl.][ ]{#2}}% (trailing "%" ensure that the line break is not converted to a space)
  {\footcite[Vgl.][#1]{#2}}%
}
% define command for quickly creating a direct quote citation
\newcommand{\directcite}[2][]{
  \ifthenelse{\equal{#1}{}}
  {\footcite[ ]{#2}}%
  {\footcite[#1]{#2}}%
}

% create captions where the text on the page ends with a footnotemark
\newcommand{\captionwithfootnotemark}[1]{\caption[#1]{#1\footnotemark}}

%define commands to quickly reference tables, figures and appendixes
\newcommand{\abb}[1]{\autoref{fig:#1}}
\newcommand{\tab}[1]{\autoref{tab:#1}}
\newcommand{\anh}[1]{\hyperref[sec:A#1]{Anhang A#1}}

% define command for quickly including an image with a custom footnote
\newcommand{\listedimage}[5][]{%
  \vspace{1em}%
  \begin{figure}[H]
    \centering
    \ifthenelse{\equal{#1}{}}
      {\includegraphics[width=0.6\textwidth]{#2}}%
      {\includegraphics[width=#1\textwidth]{#2}}%
    \captionwithfootnotemark{#4}
    \label{fig:#3}
  \end{figure}
  \vspace{-1em}%
  \footnotetext{#5}
}

% define command for quickly including a table with a custom footnote
\newcommand{\listedtable}[4]{%
  \vspace{1em}%
  \begin{table}[H]
    \centering
    \small
    #1
    \captionwithfootnotemark{#3}
    \label{tab:#2}
  \end{table}
  \vspace{-1em}%
  \footnotetext{#4}
}

% define command for numbering and displaying interview questions
\newcounter{interviewquestion}
\newlength{\questionwidth}
\setlength{\questionwidth}{\linewidth}
\addtolength{\questionwidth}{-1.4em}
\newcommand{\question}[1]{%
    \par
    \setstretch{1.4}
    \vspace{1em}
    \textbf{\selectfont\fontfamily{ptm}\selectfont\theinterviewquestion.\hspace{0.5em}\parbox[t]{\questionwidth}{#1}\nopagebreak}%
    \stepcounter{interviewquestion}
    \par
    \setstretch{1.5}
}

% Appendix (Anhang)
\newcommand{\listappendixname}{Anhangsverzeichnis}% name (resulting in headline) is optional
\newlistof{appendices}{app}{\listappendixname}
% counter for Anhang-Subsections
\newcounter{anhangsubsec}
\renewcommand{\theanhangsubsec}{\arabic{anhangsubsec}}
% command to create Anhang-Subsections
\newcommand{\appendixheadline}[1]{
  \stepcounter{anhangsubsec}
  \setcounter{interviewquestion}{1}
  \subsection*{A\theanhangsubsec\hspace{0.5em}\parbox[t]{13cm}{\noexpandedacronyms{#1}\\ }}% asterisk to suppress automatic numbering
  \label{sec:A\theanhangsubsec}
  %\addcontentsline{toc}{subsection}{\protect\numberline{A\theanhangsubsec} #1}% add subsection headline to table of contents
  \addcontentsline{app}{subsection}{\protect\numberline{A\theanhangsubsec}\hspace{0.1em}#1}
}

% do not force all footcite texts to end with a dot
\renewcommand{\bibfootnotewrapper}[1]{\bibsentence#1\addspace}
% place a colon after the name of the author or organisation
\renewcommand{\labelnamepunct}{\addcolon\space}
% improve the german bibliography strings for online references
\DefineBibliographyStrings{ngerman}{retrieved={Abgerufen am}, from={von}, nodate={o.J.},}

% define german name for figure labels
\renewcommand{\cftfigpresnum}{Abbildung }
\settowidth{\cftfignumwidth}{Abbildung 10\quad}
% define german name for table labels
\renewcommand{\cfttabpresnum}{Tabelle }
\settowidth{\cfttabnumwidth}{Tabelle 10\quad}

% create a counter for the numbering in Verzeichnis headlines
\newcounter{verzeichnisnummer}
\renewcommand{\theverzeichnisnummer}{\Roman{verzeichnisnummer}\ }% Counter representation
\setcounter{verzeichnisnummer}{1}% start value

% implement counter in the headlines
\addto\captionsngerman{\renewcommand{\listfigurename}{\theverzeichnisnummer Abbildungsverzeichnis}}
\newcommand{\showlistoffigures}{
  \listoffigures
  \stepcounter{verzeichnisnummer}
  \newpage
}
\addto\captionsngerman{\renewcommand{\listtablename}{\theverzeichnisnummer Tabellenverzeichnis}}
\newcommand{\showlistoftables}{
  \listoftables
  \stepcounter{verzeichnisnummer}
  \newpage
}
\newcommand{\bib}{
  \newpage
  \printbibliography[heading=bibintoc, title={\theverzeichnisnummer Literaturverzeichnis}]
  \newpage
}

% make sure that \ac{} acronyms are not expanded in the table of contents, list of figures, and list of tables
\DeclareRobustCommand{\noexpandedacronyms}[1]{{\renewcommand{\ac}[1]{\acs{##1}}#1}}
\let\oldcontentsline\contentsline
\renewcommand{\contentsline}[4]{\oldcontentsline{#1}{\noexpandedacronyms{#2}}{#3}{#4}}

% command for mentioning an acronym for the first time using \ac{}, but using a different spelling in the long form
\let\defaultac\ac%
\renewcommand{\ac}[2][]{%
  \ifthenelse{\equal{#1}{}}%
  {\defaultac{#2}}%
  {#1 (\acsu{#2})\label{acro:#2}}%
}


% styling

\definecolor{ba-blau}{HTML}{093a80}
\definecolor{gray}{HTML}{646973}
\definecolor{darkgreen}{HTML}{408335}
\definecolor{mauve}{HTML}{A9455D}
\definecolor{blue}{HTML}{1455C0}

% styling for code blocks
% please note that PDF should not be used to distribute code meant for anything but reading
\lstset{
  aboveskip=3mm,
  belowskip=3mm,
  showstringspaces=false,
  columns=flexible,
  basicstyle={\small\ttfamily},
  keywordstyle=\color{blue},
  commentstyle=\color{darkgreen},
  stringstyle=\color{mauve},
  numberstyle=\tiny\color{gray},
  numbers=left,
  breaklines=true,
  %breakatwhitespace=true, % nur bei Leerzeichen Zeilenumbrüche einfügen
  tabsize=3,
  lineskip=1pt,
  % no lines above and below the block:
  %frame=none,
  frame=tb,
  inputencoding=utf8,
  extendedchars=true,
  literate={ä}{{\"a}}1 {ö}{{\"o}}1 {ü}{{\"u}}1 {Ä}{{\"A}}1 {Ö}{{\"O}}1 {Ü}{{\"U}}1 {ß}{{\"s}}1,
}

% set default line spacing
\setstretch{1.5}

% optional: set very high tolerance for whitespace between words before enacting automatic hyphenation (Silbentrennung)
% remove these commands for normal automatic hyphenation
\hyphenpenalty=7000
\exhyphenpenalty=7000
\tolerance=10000

% improve labels for unordered lists
\renewcommand{\labelitemii}{$\circ$}
\renewcommand{\labelitemiii}{-}

% Allow up to 4 levels of sectioning
\setcounter{secnumdepth}{4}
\setcounter{tocdepth}{4}
\newcounter{subsubsubsection}[subsubsection]
\renewcommand{\thesubsubsubsection}{\thesubsubsection.\arabic{subsubsubsection}}
% indent subsubsubsections correctly
\newlength{\subsubsubsectionwidth}
\setlength{\subsubsubsectionwidth}{\linewidth}
\addtolength{\subsubsubsectionwidth}{-3.5em}
% create subsubsubsection command
\newcommand{\subsubsubsection}[1]{%
    \refstepcounter{subsubsubsection}%
    \addcontentsline{toc}{paragraph}{\protect\numberline{\thesubsubsubsection}#1}%
    \paragraph*{\thesubsubsubsection\hspace{0.5em}\parbox[t]{\subsubsubsectionwidth}{\noexpandedacronyms{#1}\\ }\nopagebreak}%
}

% set section headings fontsizes, font family (Times New Roman) and line spacing
\sectionfont{\fontsize{14}{16.8}\selectfont\fontfamily{ptm}\selectfont}
\subsectionfont{\fontsize{12}{14.4}\selectfont\fontfamily{ptm}\selectfont}
\subsubsectionfont{\fontsize{12}{14.4}\selectfont\fontfamily{ptm}\selectfont}
\paragraphfont{\fontsize{12}{14.4}\selectfont\fontfamily{ptm}\selectfont}
% color section headings
\addtokomafont{disposition}{\color{ba-blau}}
\addtokomafont{sectionentry}{\color{ba-blau}}

% styling of the title page header and footer
\fancypagestyle{Deckblatt}{
    \pagestyle{fancy}
    \fancyhead{}
    \setlength{\headheight}{65pt}
    \setlength{\headsep}{30pt}
    \chead{\includegraphics[]{deckbild.jpeg} \\}
    %page footer
    \fancyfoot{}% clear footer
}
% styling of the default page header and footer
\fancypagestyle{defaultPageStyle}{
    \pagestyle{fancy}
    \fancyhead{}
    \setlength{\headheight}{65pt}
    \setlength{\headsep}{30pt}
    \chead{\includegraphics[]{deckbild.jpeg} \\}
    %page footer
    \fancyfoot{}
    \fancyfoot[R]{\thepage}% place page numbers in the lower right corner
    %\fancyfoot[LE,RO]{\thepage}% or use this command to alternate the page number position. See README.md for more info first.
}
% color horizontal lines in header and footer
\renewcommand{\headrule}{\color{gray}\hrule width\headwidth height\headrulewidth \vskip-\headrulewidth}
\renewcommand{\footrule}{\color{gray}\hrule width\headwidth height\footrulewidth \vskip-\footrulewidth}
\renewcommand{\headrulewidth}{0.4pt}

% styling of the document title
\newcommand{\documenttitle}[1]{{\fontsize{20pt}{24pt}\selectfont\textbf{\color{ba-blau}#1}}}
% styling of the document subtitle
\newcommand{\documentsubtitle}[1]{{\fontsize{14pt}{17pt}\selectfont\textbf{\color{ba-blau}#1}}}
% styling of non-chapter page titles
\newcommand{\pagetitle}[1]{{\fontsize{14pt}{17pt}\selectfont\fontfamily{ptm}\textbf{\color{ba-blau}#1}}\\ [1em]}

% place the dots that lead to the page numbers in the table of contents
\renewcommand{\cftsecleader}{\cftdotfill{\cftdotsep}} % for sections
\renewcommand{\cftsubsecleader}{\cftdotfill{\cftdotsep}} % for subsections

% remove indentation from entries of the list of figures, tables and appendices
\setlength{\cftfigindent}{0pt}
\setlength{\cfttabindent}{0pt}
\let\oldloa\listofappendices
\renewcommand{\listofappendices}{\renewcommand{\cftsubsecindent}{0pt}\oldloa}

% set behaviour of some table columns (hyphenation, fontsize 10)
\newcommand{\HY}{\hyphenpenalty=25\exhyphenpenalty=25}% re-enable hyphenation locally inside the column
\renewcommand{\tabularxcolumn}[1]{>{\HY\raggedright\arraybackslash\hspace{0pt}}p{#1}}% tabularx table cells default
\newcolumntype{P}[1]{>{\HY\raggedright\arraybackslash\hspace{0pt}}p{#1}}% define tabular P table cell

% caption styling
\captionsetup{
    font=footnotesize,% use fontsize 10pt
    justification=centering,% center captions horizontally
    width=0.8\linewidth,% only span max. 80% of the width of a text line
    format=plain,% do not hangindent lines from the label
    labelfont=bf,% print the label in bold
}

% add some spacing after the number in the footnote
\let\oldfootnote\footnote
\renewcommand{\footnote}[1]{\oldfootnote{\hspace{0.2em}#1}}
\let\oldfootnotetext\footnotetext
\renewcommand{\footnotetext}[1]{\oldfootnotetext{\hspace{0.2em}#1}}


% start of document

\begin{document}

% use different pagestyling
\thispagestyle{Deckblatt}
% ignore this page when numbering
\pagenumbering{gobble}

% title page
\include{sections/nebeninhalte/deckblatt}

% use default page styling from now on
\pagestyle{defaultPageStyle}

\include{sections/nebeninhalte/vorwort}
\include{sections/nebeninhalte/sperrvermerk}
\include{sections/nebeninhalte/gleichbehandlung}

% start page numbering in roman numerals
\pagenumbering{Roman}
\setcounter{page}{1}
\renewcommand{\footrulewidth}{0.4pt}

\tableofcontents
\newpage

\showlistoffigures % Abbildungsverzeichnis - wenn leer, auskommentieren

\showlistoftables % Tabellenverzeichnis - wenn leer, auskommentieren

\include{sections/nebeninhalte/abkuerzungsverzeichnis}

% store last roman page number
\newcounter{preamblecounterstate}
\setcounter{preamblecounterstate}{\value{page}}

% use arabic numbers for actual content pages
\pagenumbering{arabic}

% include the text section pages located in the given folder
\section{Demo-Seite}
Auf dieser Seite befinden sich Umsetzungsbeispiele für häufig benötigte Elemente, wie beispielsweise:
\begin{itemize}
    \item ungeordnete Listen
    \item mit mehreren Elementen
\end{itemize}
\begin{enumerate}
    \item geordnete Listen
    \item und vieles mehr
\end{enumerate}

% einfach Tabellen erstellen: https://github.com/Schlump02/barm-latex-vorlagen/wiki/N%C3%BCtzliche-Links-und-Erweiterungen#einfach-tabellen-erstellen
\listedtable{
    \begin{tabular}{|P{4cm}|P{3cm}|}% | = senkrechte Linie, P{4cm} = linksbündige Zeile mit 3cm Breite, | = weitere senkrechte Linie etc.
        \hline
        \textbf{Angebot} & \textbf{Preis} \\% fette Texte mit \textbf{...}
        \hline
        Kebab & 7 € \\% Zeilen werden mit & unterteilt und enden mit \\
        \hline
        Adana & 6 € \\
        \hline % horizontale Linie
        Dönerteller & 10 € \\
        \hline
        Mercimek Suppe & 3 € \\
        \hline
    \end{tabular}
}{preistabelle}{Beispiel für eine einfache Tabelle}{Vgl. \cite[10]{DemoQuelle}}
% Kurzbeschriftung (wichtig für den \tab{}-Befehl, siehe weiter unten), Beschriftung, Fußnote


\listedtable{% Tabelle zu Tabellenverzeichnis hinzufügen
    \begin{tabularx}{0.5\textwidth}{X|X}% 0.5\textwidth = Tabelle hat 50% der Breite einer Zeile (100%=15,5cm), X = Spalte mit dynamischer Breite
        \rowcolor[HTML]{D7DCE1}
        \textbf{Angebot} & \textbf{Preis} \\
        \rowcolor[HTML]{F0F3F5}
        Kebab & 7 € \\
        \rowcolor[HTML]{F8F9F9}% Hexcode der Zeilenhintergrundfarbe
        Adana & 6 € \\
        \rowcolor[HTML]{F0F3F5}
        Dönerteller & 10 € \\
        \rowcolor[HTML]{F8F9F9}
        Mercimek Suppe & \cellcolor[HTML]{C9EB9E} 3 € \\% Zellenfarbe verändert
    \end{tabularx}
}{preistabelle_farbig}{Eine Tabelle mit angepassten Hintergrundfarben}{adaptiert aus \cite[10]{DemoQuelle}}
% Kurzbeschriftung (wichtig für den \tab{}-Befehl, siehe weiter unten), Beschriftung, Fußnote

Die Tabelle zeigt den Preis eines Dönertellers. Dieser lässt sich wie folgt berechnen:
\begin{equation}
    15 = \sum_{n=1}^{10} \frac{n}{20} + \sum_{k=1}^{5} \frac{2k}{10} - \sum_{i=1}^{3} i
\end{equation}% Gleichungen müssen nicht ins Abbildungsverzeichnis

Der folgende Abschnitt könnte hilfreich für eine Ausarbeitung in der Informatik sein.
\begin{figure}[H] % Abschnitt als Abbildung kennzeichnen (H = hier im Text platzieren)
    \begin{lstlisting}[language=python]
        # say hi
        def hello_world():
            print("Hello, World!")
    \end{lstlisting}
    \captionwithfootnotemark{Ausschnitt aus main.py}% Die (verpflichtende) Quellenangabe hierzu wird durch den \footnotetext-Befehl weiter unten gesetzt
    \label{fig:meincode}% wichtig für die Benutzung mit \abb{} (siehe weiter unten)
\end{figure}
\footnotetext{Quelle: eigene Erstellung}
%\footnotetext{\cite[10]{DemoQuelle}} falls es eine Quelle aus der Bib ist

% Abbildung einfügen
% Breite, Dateiname, Kurzbeschriftung, Beschriftung, Fußnote mit Quelle aus der Bib
\listedimage[0.6]{deckbild.jpeg}{logo}{Das Logo der BA}{adaptiert aus \cite[6-7]{theisen2011}}
% Die Breite wird relativ zur Zeilenlänge angepasst, der Wert zwischen 0.5 entspricht also der halben Zeilenlänge

Auf \abb{logo} und \tab{preistabelle} kann im Text verwiesen werden.% Hierfür ist die Kurzbeschriftung notwendig.

\subsection{Zitierbeispiele}

\subsubsection{Direkte und indirekte Zitate}

Nach indirekten Zitaten steht die Fußnotenzahl hinter dem Punkt.\indirectcite[10-11]{theisen2011}\\% Indirekte (sinngemäße) Zitate geben den Inhalt der Quelle in eigenen Worten wieder.
Bei "direkten Zitaten"\directcite[12]{theisen2011} steht sie "[\ldots] direkt nach den Anführungszeichen."\directcite[13]{theisen2011}\\% Direkte Zitate geben den Wortlaut der Quelle wieder. Sie müssen in Anführungszeichen gesetzt werden.
% Für Zitate in Zitaten (einzelne Anführungszeichen): ,Zitat`
Anführungszeichen sollten nur für wörtliche Zitate genutzt werden; Hervorhebungen sollten \textbf{fett} oder \textit{kursiv} formatiert sein.

\newpage
\subsubsection{Beispiel für jeden Quelltyp}
% "Vgl." steht nur in Fußnoten für indirekte (nicht wörtliche) Zitate.
% Es genügt den Anforderungen, Kurzbelege zu setzen, die auf den Vollbeleg im Literaturverzeichnis verweisen.
% Wenn eine fremdsprachige Aussage ins Deutsche übersetzt wird, stellt das eine Paraphrase dar und ist somit ein indirektes Zitat.

Buch/Monografie\indirectcite[40-58]{theisen2011}\\
Sammelwerk\indirectcite[204-205]{maier2004}\\
Zeitschriften-/Journalartikel\indirectcite[10-12]{chodorowreich2022loan}\\
Zeitungsartikel\indirectcite[12-15]{dick2012neugierige}\\
Internet\indirectcite{capital2014}\\
Gesetzestext\footnote{Vgl. §433 Abs. 1 Satz 1 BGB}\nocite{bgb}\\
Gerichtsurteil\indirectcite[460]{bverfgh1968}\\
öffentliches Dokument\indirectcite[12-15]{eu2022access}\\
internes Dokument\indirectcite[12-15]{abcorganigramm}\\% muss auf separatem Datenträger beigefügt werden. Am besten sammelt man schon frühzeitig alle Dokumente in einem Ordner.
(unvollständige Quellenangaben)\indirectcite{blankmaier}

\subsubsubsection{Beispiel für eine sub-sub-sub-Überschrift}
% Diese Vorlage unterstützt (technisch und moralisch) nur max. vierfache Untergliederung
% Informationen zur mehr als vierfachen Untergliederung: https://github.com/Schlump02/barm-latex-vorlagen/wiki/Weitere-Formatierungsm%C3%B6glichkeiten#mehr-als-vierfache-untergliederung

\include{sections/textinhalte/einleitung}
\include{sections/textinhalte/kapitelvorlage}
\include{sections/textinhalte/fazit}

\begin{appendices}% Anhang - wenn leer, auskommentieren (bis inklusive \end{appendices})
  \input{sections/nebeninhalte/anhang/anhang}
\end{appendices}

% set page numbering to roman for bibliography
\pagenumbering{Roman}
% read counter stored earlier
\setcounter{page}{\value{preamblecounterstate}}

% Literaturverzeichnis
\bib

\pagenumbering{gobble}% remove page numbering
\renewcommand{\footrulewidth}{0pt}

\include{sections/nebeninhalte/ehrenwort}

\end{document}