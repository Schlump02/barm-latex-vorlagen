\newsection{Inhalte}
% Die genaue Darstellung der Inhalte kann von diesem Beispiel abweichen. Der Abschnitt sollte etwa 1-2 Seiten lang sein.
% Abbildungen und Tabellen sollten immer einen Untertitel und ggf. eine Quelle besitzen. (wie beim \image Befehl in der demoSection.tex)

\underline{Aspekt 1}
\begin{itemize}
  \item Unterpunkt 1
  \item Unterpunkt 2
\end{itemize}\vspace{10pt}

\underline{Aspekt 2}
\begin{itemize}
  \item Unterpunkt 1
  \item Unterpunkt 2
\end{itemize}\vspace{10pt}

% Breite, Dateiname, Kurzbeschriftung, Beschriftung
\image[0.5]{deckbild.jpeg}{logo}{Das Logo der BA}
% Die Breite wird relativ zur Zeilenlänge angepasst, der Wert zwischen 0.5 entspricht also der halben Zeilenlänge

Auf \abb{logo}, \tab{preistabelle} und \anh{2} kann im Text verwiesen werden.


\newpage