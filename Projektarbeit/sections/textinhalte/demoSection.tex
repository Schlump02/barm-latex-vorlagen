\section{Demo-Seite}
Auf dieser Seite befinden sich Umsetzungsbeispiele für häufig benötigte Elemente, wie beispielsweise:
\begin{itemize}
    \item ungeordnete Listen
    \item mit mehreren Elementen
\end{itemize}
\begin{enumerate}
    \item geordnete Listen
    \item und vieles mehr
\end{enumerate}

% einfach Tabellen erstellen: https://github.com/Schlump02/barm-latex-vorlagen/wiki/N%C3%BCtzliche-Links-und-Erweiterungen#einfach-tabellen-erstellen
\listedtable{
    \begin{tabular}{|P{4cm}|P{3cm}|}% | = senkrechte Linie, P{4cm} = linksbündige Zeile mit 3cm Breite, | = weitere senkrechte Linie etc.
        \hline
        \textbf{Angebot} & \textbf{Preis} \\% fette Texte mit \textbf{...}
        \hline
        Kebab & 7 € \\% Zeilen werden mit & unterteilt und enden mit \\
        \hline
        Adana & 6 € \\
        \hline % horizontale Linie
        Dönerteller & 10 € \\
        \hline
        Mercimek Suppe & 3 € \\
        \hline
    \end{tabular}
}{preistabelle}{Beispiel für eine einfache Tabelle}{Vgl. \cite[10]{DemoQuelle}}
% Kurzbeschriftung (wichtig für den \tab{}-Befehl, siehe weiter unten), Beschriftung, Fußnote


\listedtable{% Tabelle zu Tabellenverzeichnis hinzufügen
    \begin{tabularx}{0.5\textwidth}{X|X}% 0.5\textwidth = Tabelle hat 50% der Breite einer Zeile (100%=15,5cm), X = Spalte mit dynamischer Breite
        \rowcolor[HTML]{D7DCE1}
        \textbf{Angebot} & \textbf{Preis} \\
        \rowcolor[HTML]{F0F3F5}
        Kebab & 7 € \\
        \rowcolor[HTML]{F8F9F9}% Hexcode der Zeilenhintergrundfarbe
        Adana & 6 € \\
        \rowcolor[HTML]{F0F3F5}
        Dönerteller & 10 € \\
        \rowcolor[HTML]{F8F9F9}
        Mercimek Suppe & \cellcolor[HTML]{C9EB9E} 3 € \\% Zellenfarbe verändert
    \end{tabularx}
}{preistabelle_farbig}{Eine Tabelle mit angepassten Hintergrundfarben}{adaptiert aus \cite[10]{DemoQuelle}}
% Kurzbeschriftung (wichtig für den \tab{}-Befehl, siehe weiter unten), Beschriftung, Fußnote

Die Tabelle zeigt den Preis eines Dönertellers. Dieser lässt sich wie folgt berechnen:
\begin{equation}
    15 = \sum_{n=1}^{10} \frac{n}{20} + \sum_{k=1}^{5} \frac{2k}{10} - \sum_{i=1}^{3} i
\end{equation}% Gleichungen müssen nicht ins Abbildungsverzeichnis

Der folgende Abschnitt könnte hilfreich für eine Ausarbeitung in der Informatik sein.
\begin{figure}[H] % Abschnitt als Abbildung kennzeichnen (H = hier im Text platzieren)
    \begin{lstlisting}[language=python]
        # say hi
        def hello_world():
            print("Hello, World!")
    \end{lstlisting}
    \captionwithfootnotemark{Ausschnitt aus main.py}% Die (verpflichtende) Quellenangabe hierzu wird durch den \footnotetext-Befehl weiter unten gesetzt
    \label{fig:meincode}% wichtig für die Benutzung mit \abb{} (siehe weiter unten)
\end{figure}
\footnotetext{Quelle: eigene Erstellung}
%\footnotetext{\cite[10]{DemoQuelle}} falls es eine Quelle aus der Bib ist

% Abbildung einfügen
% Breite, Dateiname, Kurzbeschriftung, Beschriftung, Fußnote mit Quelle aus der Bib
\listedimage[0.6]{deckbild.jpeg}{logo}{Das Logo der BA}{adaptiert aus \cite[6-7]{theisen2011}}
% Die Breite wird relativ zur Zeilenlänge angepasst, der Wert zwischen 0.5 entspricht also der halben Zeilenlänge

Auf \abb{logo} und \tab{preistabelle} kann im Text verwiesen werden.% Hierfür ist die Kurzbeschriftung notwendig.

\subsection{Zitierbeispiele}

\subsubsection{Direkte und indirekte Zitate}

Nach indirekten Zitaten steht die Fußnotenzahl hinter dem Punkt.\indirectcite[10-11]{theisen2011}\\% Indirekte (sinngemäße) Zitate geben den Inhalt der Quelle in eigenen Worten wieder.
Bei "direkten Zitaten"\directcite[12]{theisen2011} steht sie "[\ldots] direkt nach den Anführungszeichen."\directcite[13]{theisen2011}\\% Direkte Zitate geben den Wortlaut der Quelle wieder. Sie müssen in Anführungszeichen gesetzt werden.
% Für Zitate in Zitaten (einzelne Anführungszeichen): ,Zitat`
Anführungszeichen sollten nur für wörtliche Zitate genutzt werden; Hervorhebungen sollten \textbf{fett} oder \textit{kursiv} formatiert sein.

\newpage
\subsubsection{Beispiel für jeden Quelltyp}
% "Vgl." steht nur in Fußnoten für indirekte (nicht wörtliche) Zitate.
% Es genügt den Anforderungen, Kurzbelege zu setzen, die auf den Vollbeleg im Literaturverzeichnis verweisen.
% Wenn eine fremdsprachige Aussage ins Deutsche übersetzt wird, stellt das eine Paraphrase dar und ist somit ein indirektes Zitat.

Buch/Monografie\indirectcite[40-58]{theisen2011}\\
Sammelwerk\indirectcite[204-205]{maier2004}\\
Zeitschriften-/Journalartikel\indirectcite[10-12]{chodorowreich2022loan}\\
Zeitungsartikel\indirectcite[12-15]{dick2012neugierige}\\
Internet\indirectcite{capital2014}\\
Gesetzestext\footnote{Vgl. §433 Abs. 1 Satz 1 BGB}\nocite{bgb}\\
Gerichtsurteil\indirectcite[460]{bverfgh1968}\\
öffentliches Dokument\indirectcite[12-15]{eu2022access}\\
internes Dokument\indirectcite[12-15]{abcorganigramm}\\% muss auf separatem Datenträger beigefügt werden. Am besten sammelt man schon frühzeitig alle Dokumente in einem Ordner.
(unvollständige Quellenangaben)\indirectcite{blankmaier}

\subsubsubsection{Beispiel für eine sub-sub-sub-Überschrift}
% Diese Vorlage unterstützt (technisch und moralisch) nur max. vierfache Untergliederung
% Informationen zur mehr als vierfachen Untergliederung: https://github.com/Schlump02/barm-latex-vorlagen/wiki/Weitere-Formatierungsm%C3%B6glichkeiten#mehr-als-vierfache-untergliederung
